\documentclass[a4paper, 12pt]{scrartcl}
\usepackage[
    typ=ueb,
    fach=Mathematik,
    lerngruppe=9/6,
    loesungen=keine,
    module={Symbole,Bewertung},
    datumAnzeigen,
  namensfeldAnzeigen,
]{schule}

\author{Marcel Lehmann}
\date{\today}
\title{}

\usepackage{babel,fouriernc,parskip,booktabs,array}
\usepackage{enumitem}

\usepackage[a4paper,inner=3cm,lmargin=2.5cm,outer=4cm,
tmargin=2.5cm,bmargin=2.5cm]{geometry}

\usepackage{pgfpages}                % <— load the package
  \pgfpagesuselayout{2 on 1}[a4paper, landscape,border shrink=5mm] % <— set options
\usepackage{atbegshi}
  % duplicate the content at shipout time
  \AtBeginShipout{
    \pgfpagesshipoutlogicalpage{1}\copy\AtBeginShipoutBox
    \pgfpagesshipoutlogicalpage{2}\box\AtBeginShipoutBox
    \pgfshipoutphysicalpage
  }
\setlist{nolistsep}
\renewcommand\labelitemi{--}



\begin{document}
\begin{center}
 \large{\textbf{Tägliche Übung}}
\end{center}




\setzeAufgabentemplate{schule-randpunkte}

\begin{aufgabe}[points = 3]
 \textbf{Operator} und die Aufgabe.
\end{aufgabe}


\begin{aufgabe}
 \textbf{Operator} und die Aufgabe.
 \begin{teilaufgaben}
  \teilaufgabe[2] xxx
  \teilaufgabe[3] xxx
 \end{teilaufgaben}
\end{aufgabe}

\begin{aufgabe}[points = 3]
 \textbf{Operator} und die Aufgabe.
\end{aufgabe}


\begin{aufgabe}
 \textbf{Operator} und die Aufgabe.
 \begin{teilaufgaben}
  \teilaufgabe[2] xxx
  \teilaufgabe[3] xxx
 \end{teilaufgaben}
\end{aufgabe}


\begin{center}
 \achtung{TÜ wird zufällig eingesammelt!} \\
\end{center}

\punktuebersicht*



\end{document}
