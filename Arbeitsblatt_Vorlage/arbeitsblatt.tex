\documentclass[a4paper,10pt]{article}

\usepackage{ucs}
\usepackage[utf8x]{inputenc}
\usepackage[T1]{fontenc}
\usepackage[ngerman]{babel}
\usepackage{amsmath, amssymb, amstext}
\usepackage{graphicx}
\usepackage[german]{layout}
\usepackage{hhline}
\usepackage{tabularx}

\usepackage{pgfplots}
\pgfplotsset{compat=newest, axis lines=middle}% siehe pgfplots-Anleitung

\usepackage[a4paper,inner=2.5cm,lmargin=2.5cm,outer=2.5cm,
tmargin=2.5cm,bmargin=2.5cm]{geometry}

\usepackage[headsepline, footsepline]{scrlayer-scrpage}
    \pagestyle{scrheadings}

    \ihead{\today}
    \chead{Potenzfunktionen}
    \ohead{Klasse 9}

    \ifoot{© 2023 by Marcel Lehmann is licensed under CC BY-NC-SA 4.0 }
    \cfoot{}
    \ofoot{
        \includegraphics[width = 0.06\textwidth]{cc}
        }


\begin{document}
    %\layout
    \begin{center}
        \textbf{\Large{Arbeitsblatt: Extremwertaufgaben}}
        %\hrule
    \end{center}

    \subsection*{1. Allgemeine Probleme}
    Es stehen für Extremalaufgaben drei Kategorien zur Verfügung:
    \begin{enumerate}
    \item Maximierung
    \item Minimierung
    \item Distanz
        \begin{enumerate}
        \item  x - y < 0
        \item x - y > 0
        \end{enumerate}
    \end{enumerate}

    \subsection*{2. Tabellenaufgabe}
    \textbf{Positioniere} dich und \textbf{korrigiere} falsche Aussagen!\\
        \begin{tabularx}{1\linewidth}{|X|p{14.5cm}|}
        \hline
                    & \textbf{Behauptungen}
        \\ \hline
        \textbf{a)} & Das ist eine Behauptung.....\\&\\&\\&\\&\\ &
        \\ \hline
        \textbf{b)} & Das ist eine Behauptung.....\\&\\&\\&\\&\\ &
        \\ \hline
        \end{tabularx}

\subsection*{3. Tabelle mit 2 Spalten}
     \begin{tabularx}{\linewidth}{|X|X|}
        \hline
        {
            Text links
        }
        &
        {
            Text rechts
        }
        \\ \hline
    \end{tabularx}
\subsection*{4. Bilder}
    \includegraphics[width = 0.5\textwidth]{cc}
\subsection*{5. Axis}
\begin{center}
            \begin{tikzpicture}
                \begin{axis}[
                    %Farbe und Transparenz vom Grid
                    grid style={grey!5},
                    xmin=-4,xmax=4,
                    ymin=-4,ymax=4,
                    x=1cm,
                    y=1cm,
                    minor tick num=2,% zwischen zwei Haupt-Achsmarkierungen jeweils einen Untermarkierungen einfügen
                    minor tick length=0pt,% Länge der Untermarkierungen auf 0 setzen, sie also unterdrücken
                    grid={both},% Gitterlinien sowohl durch die Positionen der Haupt- als auch Untermarkierungen
                    xlabel={x}, %x-achsenbeschriftung (BITTE NICHT DEN MATHEMODUS MISSBRAUCHEN)
                    ylabel={y}, %y-achsenbeschriftung (BITTE NICHT DEN MATHEMODUS MISSBRAUCHEN)
                    label style={font=\itshape}% Achsenbeschriftung kursiv
                    ]
                    \addplot[domain=-3:3] {x}  node[right,pos=0.1]{$f(x)=x$};% Lineare Funktion
                    \addplot[domain=-3:3,samples=1000] {0.5*x^2} node[right,pos=0.1]{$f(x)=x^{2}$};% Quadratische Funktion
                \end{axis}
                \end{tikzpicture}
        \end{center}
\end{document}
